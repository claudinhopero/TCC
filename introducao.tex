% ----------------------------------------------------------
% Introdução (exemplo de capítulo sem numeração, mas presente no Sumário)
% ----------------------------------------------------------
\chapter[Introdução]{Introdução}
%\addcontentsline{toc}{chapter}{Introdução}
% ----------------------------------------------------------

Com a crescente demanda de soluções tecnológicas para resolução de problemas cotidianos governamentais, o conceito de \textbf{cidades inteligentes} se torna cada vez mais necessário na gestão de políticas públicas. Desde 2001, o Departamento das Nações Unidas para Assuntos Econômicos e Sociais (DESA) publica, quase bi-anualmente, o \textit{Estudo de e-government das Nações Unidas}, ranqueando os países com maior aplicação prática de \textbf{e-government}.

O crescimento de dados de geolocalização, gerado pela crescente aderência populacional ao uso de \textit{smartphone}, criou um ambiente propício para o aumento de aplicativos \textbf{e-government}. Atualmente os suporte de serviços \textbf{e-government} são aplicados em diferentes áreas como prevenção de desastres, meio ambiente, políticas sociais, em diferentes países como Brasil, Áustria, Índia, por exemplo.

Este trabalho focará no desenvolvimento de um algoritmo, com o objetivo de identificar o agrupamento de trajetórias com comportamento similares. As trajetórias representam a movimentação de usuários de transporte público quando desembarcados.